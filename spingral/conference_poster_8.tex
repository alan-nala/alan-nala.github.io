%%%%%%%%%%%%%%%%%%%%%%%%%%%%%%%%%%%%%%%%%
% baposter Portrait Poster
% LaTeX Template
% Version 1.0 (15/5/13)
%
% Created by:
% Brian Amberg (baposter@brian-amberg.de)
%
% This template has been downloaded from:
% http://www.LaTeXTemplates.com
%
% License:
% CC BY-NC-SA 3.0 (http://creativecommons.org/licenses/by-nc-sa/3.0/)
%
%%%%%%%%%%%%%%%%%%%%%%%%%%%%%%%%%%%%%%%%%

%----------------------------------------------------------------------------------------
%	PACKAGES AND OTHER DOCUMENT CONFIGURATIONS
%----------------------------------------------------------------------------------------

\documentclass[a0paper,portrait]{baposter}

\usepackage[spanish]{babel}
\usepackage[latin1]{inputenc}
%\usepackage{tikz}
\usepackage{color}
\usepackage{graphicx}
\usepackage[font=small,labelfont=bf]{caption} % Required for specifying captions to tables and figures
\usepackage{booktabs} % Horizontal rules in tables
\usepackage{relsize} % Used for making text smaller in some places

\graphicspath{{figures/}} % Directory in which figures are stored

\definecolor{bordercol}{RGB}{40,40,40} % Border color of content boxes
\definecolor{headercol1}{RGB}{186,215,230} % Background color for the header in the content boxes (left side)
\definecolor{headercol1}{RGB}{100,255,255} % Background color for the header in the content boxes (left side)
\definecolor{headercol2}{RGB}{80,80,80} % Background color for the header in the content boxes (right side)
\definecolor{headerfontcol}{RGB}{0,0,0} % Text color for the header text in the content %boxes
\definecolor{boxcolor}{RGB}{186,215,230} % Background color for the content in the content boxes
\definecolor{boxcolor}{RGB}{245,255,255} % Background color for the content in the content boxes
\definecolor{boxcolor}{RGB}{255,255,255} % Background color for the content in the content boxes

\begin{document}

\background{ % Set the background to an image (background.pdf)
\begin{tikzpicture}[remember picture,overlay]
\draw (current page.north west)+(-2em,2em) node[anchor=north west]
{\includegraphics[height=.01\textheight]{background}};
\end{tikzpicture}
}

\begin{poster}{
grid=false,
borderColor=bordercol, % Border color of content boxes
headerColorOne=headercol1, % Background color for the header in the content boxes (left side)
headerColorTwo=headercol2, % Background color for the header in the content boxes (right side)
headerFontColor=headerfontcol, % Text color for the header text in the content boxes
boxColorOne=boxcolor, % Background color for the content in the content boxes
headershape=roundedright, % Specify the rounded corner in the content box headers
headerfont=\Large\sf\bf, % Font modifiers for the text in the content box headers
textborder=rectangle,
background=user,
headerborder=open, % Change to closed for a line under the content box headers
boxshade=plain
}
{}
%
%----------------------------------------------------------------------------------------
%	TITLE AND AUTHOR NAME
%----------------------------------------------------------------------------------------
{\sf\bf\\ Entrelazamiento en cadenas dimerizadas\\ \vspace{.25em} de espin $s$ } % Poster title
{\vspace{.5em} A.\ Boette, R.\ Rossignoli, N. Canosa, J.M.\ Matera\\ % AuthS$or names
{\vspace{.25em}  IFLP/Depto.\ de F\'{\i}sica-CONICET-CIC,
	Universidad Nacional de La Plata}}
%j.smith@uni.edu, j.smith2@uni.edu, j.smith3@uni.edu}} % Author email addresses
{\includegraphics*[scale=0.25]{unlp.eps}} % University/lab logo

%----------------------------------------------------------------------------------------
%	INTRODUCCION
%----------------------------------------------------------------------------------------

\headerbox{Resumen}{name=introduction,column=0,row=0}{

\smaller Se analiza el entrelazamiento de pares en el estado fundamental de cadenas dimerizadas de espin $s$ con interacci\'on $XY$,
inmersas en un campo magn\'etico, a partir  de una aproximaci\'on de campo medio autoconsistente  basada en pares. Se muestra el surgimiento de sucesivas
fases dimerizadas al aumentar el campo, que originan ``plateaus'' de magnetizaci\'on y entrelazamiento de par, separadas por fases con ruptura de simetr\'{\i}a de
paridad de espin y bajo entrelazamiento del par.  Los resultados muestran un buen acuerdo con los exactos y difieren sustancialmente de los predichos por el  campo
medio convencional. Se analiza tambi\'en el l\'{\i}mite de alto espin [1,2].
}

%----------------------------------------------------------------------------------------
%	MATERIALS AND METHODS
%----------------------------------------------------------------------------------------

\headerbox{Modelo y M\'etodo}{name=methods,column=0,below=introduction}{
\vspace*{.4cm}
\begin{center}
\hspace*{0.75cm}{\includegraphics*[trim={0cm 0cm 0cm 24cm},clip,scale=.2]{fig1a}}
\end{center}
\vspace*{-.65cm}
%\begin{itemize}
%\item

\hspace*{.25cm}{\bf Hamiltoniano}
%\end{itemize}
{\color{blue}
{\small \[
H=\sum_{i\; {\rm par}}\! B(s^z_{i-1}+s^z_{i})-\sum_{\mu}J_\mu(s_{i-1}^\mu
s_{i}^\mu+\alpha s_{i}^\mu s_{i+1}^\mu)\]}}
con {\color{blue} $\mu=x,y$, $\chi=J_y/J_x\leq 1$, $|\alpha|\leq 1$}.

Aproximaci\'on de campo medio de pares (GMF) basada en {\color{blue} $|\Psi\rangle=\otimes_{i\;{\rm par}} |\psi_i\rangle$}.
Hamiltoniano autoconsistente {\color{blue} $h=\sum_i h_i$} con
 {\color{blue}\begin{eqnarray}h_i&=&B(s^z_{i-1}+s^z_{i})-\sum_{\mu}J_\mu[s_{i-1}^\mu s_{i}^\mu\nonumber\\&&+\alpha(s_{i}^\mu \langle s_{i+1}^\mu\rangle+s_{i-1}^\mu \langle s_{i-2}^\mu\rangle)]\nonumber\end{eqnarray}}
Acoplamiento interno tratado en forma {\color{blue}exacta. }\\
Simetr\'ia de Paridad:  {\color{blue}$[H,P_z]=0$,  $P_z=e^{i\pi S_z}$ }\\
%\begin{itemize}
%\item
 {\color{blue} $\bullet$} Fases dimerizadas:  {\color{blue}$\langle s_i^\mu\rangle=0$, $[h,P_z]=0$}\\
 {\color{magenta}$\bullet$} Fases con simetr\'{\i}a rota: {\color{magenta} $\langle s_i^\mu\rangle\neq 0$, $[h,P_z]\neq 0$}

Condici\'on cr\'{\i}tica:
 {\color{magenta} {\small $\alpha>\left[J_x\sum_{k>0}\frac{|\langle\psi_k^0|S^x_t|\psi_0^0\rangle|^2}{E_k-E_0}\right]^{-1}$}}
Restauraci\'on de simetr\'{\i}a en fase {\color{magenta}$\bullet$} :
{\color{magenta}\[|\Psi_{\pm}\rangle\propto(1\pm P_z)|\Psi\rangle\]
\vspace*{-.5cm}}
}

%----------------------------------------------------------------------------------------
%	CONCLUSION
%----------------------------------------------------------------------------------------

\headerbox{Conclusi\'on}{name=conclusion,column=0,below=methods}{
	\hfill\break
GMF describe correctamente la magnetizaci\'on,  el entrelazamiento interno del par y el entrelazamiento  de un espin y par con el resto del sistema.  \\

Surgimiento de multiples fases dimerizadas para $\alpha$ bajo al aumentar $s$.
Originan plateaus de entrelazamiento de par y magnetizaci\'on.\\

Separadas por fases con ruptura de simetr\'{\i}a de paridad y entrelazamiento par--resto, con GS cuasi degenerado.\\
%Manifestaciones tambi\'en evidentes en espectro de energ\'{\i}as. \\

Para $s$ creciente, el entrelazamiento del par en fases dimerizadas\\
{\color{blue} $\bullet$} satura en sistemas $XY$\\
{\color{red} $\bullet$} crece como $\sqrt{s}$ en sistemas $XX$\\ %$\bullet$
%Entrelazamiento de par diverge al aumentar el espin en sistemas $XX$.
}

%----------------------------------------------------------------------------------------
%	REFERENCES
%----------------------------------------------------------------------------------------

\headerbox{Referencias}{name=references,column=0,below=conclusion}{
\smaller % Reduce the font size in this block
\hfill\break
 $[1]\!$ A.\ Boette, R.\ Rossignoli, N.\ Canosa, J.M.\ Matera \\\hspace*{0.5cm}(2016)\\
 $[2]$ A.\ Boette, R.\ Rossignoli, N.\ Canosa, J.M.\ Matera, \hspace*{0.5cm}Physical Review B 91 064428 (2015)\\
%Physical Review A B 91 064428 (2010)\\
$[3]$ N.Canosa, R.Rossignoli, J.M.Matera, PRB 81 (2010) \\
$[4]$ J.M.Matera, R.Rossignoli, N.Canosa, PRA 82 (2010)

% $[1]\!$ A. Boette, R. Rossignoli, N. Canosa, J.M. Matera, Physical Review B 94 214403 (2016)


%\renewcommand{\section}[2]{\vskip 0.05em} % Get rid of the default "References" section title
%\nocite{*} % Insert publications even if they are not cited in the poster
%a\\a\\a\\a\\a

%\bibliographystyle{unsrt}
%\bibliography{sample} % Use sample.bib as the bibliography file
}

%----------------------------------------------------------------------------------------
%	ACKNOWLEDGEMENTS
%----------------------------------------------------------------------------------------

%\headerbox{Acknowledgements}{name=acknowledgements,column=0,below=references, above=bottom}{

%\smaller % Reduce the font size in this block
%Fusce mattis tellus ac odio imperdiet lobortis. Cum sociis natoque penatibus et magnis dis parturient %montes, nascetur ridiculus mus. Phasellus commodo blandit euismod. Ut porttitor cursus magna. Mauris %adipiscing pellentesque ipsum nec facilisis. Cras ornare bibendum bibendum. Ut a elit purus, vel %adipiscing.
%}

%----------------------------------------------------------------------------------------
%	RESULTS 1
%----------------------------------------------------------------------------------------

\headerbox{Resultados 1}{name=results1,span=2,column=1,row=0}{ % To reduce this block to 1 column width, remove 'span=2'
\begin{minipage}{7.5cm}
	Se muestra primero el comportamiento magn�tico exacto del
	entrelazamiento de un par aislado de dos espines {\color{blue} $s=1$} para {\color{blue} $J_y/J_x=0.75$}, mediante la negatividad  \vspace*{-0.3cm}
		{\color{blue} \[N_{12}={\textstyle\frac{1}{2}}({\rm Tr}|\rho_{12}^{\rm t_2}|-1)\]}
		junto con la magnetizaci\'on intensiva {\color{blue} $m=\langle S_z\rangle/2$} \\
	  El estado fundamental sufre {\color{magenta} dos transiciones de paridad} (inset),
	 	la \'ultima en el {\color{blue} campo factorizante [3] $B_s=\sqrt{J_y J_x}/2$ }
	 		($j_x=J_x s(1+\alpha)$)
	 		 	 \end{minipage}
\vspace*{-4.5cm}

\begin{center}
	\hspace*{8.1cm}{\includegraphics*[trim={0cm 0cm 0cm 21cm},clip,scale=.65]{fig1}}
\end{center}
\vspace*{-.55cm}

Par de espin {\color{blue} $s$:  $2s$} transiciones de paridad ({\color{blue} $0<J_y/J_x<1$})

}

%----------------------------------------------------------------------------------------
%	RESULTS 2
%----------------------------------------------------------------------------------------

\headerbox{Resultados 2}{name=results2,span=2,column=1,below=results1,above=bottom}{ % To reduce this block to 1 column width, remove 'span=2'

Se muestran ahora resultados exactos y GMF para cadenas de espin {\color{blue}$s=1$}  y
{\color{blue}$3/2$}. Para $\alpha$ peque\~no, surgen en GMF $2s$ fases dimerizadas de paridad definida para $B<B_s$, separadas por fases con paridad rota. Se observa un buen acuerdo con los resultados exactos, tanto para la negatividad del par y la magnetizaci\'on como para las entrop\'{\i}as de entrelazamiento $S_2$ (par-resto) y $S_1$ (espin-resto).
%N\'otese que $S_2<S_1$ excepto en laproximidad de $B_s$, indicando dimerizaci\'on.
\vspace*{-1.85cm}
%------------------------------------------------

%\begin{center}
%\includegraphics*[width=.6\linewidth]{fig2}
\hspace*{-1.8cm}
\includegraphics*[scale=.45]{fig2}\hspace*{-3cm}
\vspace{-13.45cm}

\hspace*{5.8cm}\includegraphics*[scale=.4]{fig3}
\vspace*{-6.35cm}

\hspace*{-1.8cm}
\includegraphics*[scale=.45]{fig5}\hspace*{-3cm}
\vspace{-13.45cm}

\hspace*{5.8cm}\includegraphics*[scale=.4]{fig6}
\vspace*{-4.45cm}

\captionof{figure}{Diagrama de Fases de GMF (sectores coloreados: fases dimerizadas) y comparaci\'on con resultados exactos para el entrelazamiento y magnetizaci\'on, en  cadenas con $s=1$ y $s=3/2$}
\vspace*{0.5cm}

%\end{center}
%------------------------------------------------
%Aliquam ac justo lectus. Nunc ultrices aliquet purus non dictum. Nulla facilisi. Quisque %vitae urna non purus sollicitudin venenatis. Aliquam erat volutpat. Cum sociis natoque
\begin{minipage}{10.5cm} Relaci\'on entre ruptura de simetr\'{\i}a y espectro exacto de energ\'{\i}as.\\ El GS exacto exhibe {\color{magenta} $2ns$} transiciones de paridad,
confinadas en\\ los intervalos de simetr\'{\i}a rota de GMF.
\end{minipage}
 \vspace*{-8.25cm}

\hspace*{10.4cm} \includegraphics*[scale=.3]{fig4}\hspace*{-3cm}
% \vspace{-13.45cm}
%------------------------------------------------
\vspace*{-.25cm}

{\bf Comportamiento para alto espin }\\

\begin{minipage}{8.75cm} Espectro de entrelazamiento de un par de espines  $s=5$.\\
Caso {\color{blue} $XY$} (izq.):  S�lo dos estados con peso en el espectro.\\
Caso {\color{red} $XX$} (der.):  Distribuci\'on gaussiana  ($J_y=J_x$).
\vspace*{0.5cm}

Negatividad del par en funci\'on del campo  y  espin.\\
Caso {\color{blue} $XY$} (izq.): Saturaci\'on de $N_{12}$, que tiende al l\'imite bos\'onico  (finito) para $s\rightarrow\infty$ [4]. \\
Caso {\color{red} $XX$} (der.): {\color{red} $N_{12}\propto \sqrt{s}$.}
\end{minipage}
\vspace*{-5.4cm}


% - Caso $XY$ (izq.):  Sólo dos estados con peso en el espectro
% - Caso $XX$ (der.):  Distribución gaussiana  ($J_y=J_x$).

% - Negatividad del par en función del campo  y  espin.
%	- Caso $XY$ (izq.): Saturación de $N_{12}$ $\rightarrow$ tiende al límite bosónico  (finito) para $s\rightarrow\infty$ [4].
 %	- Caso $XX$ (der.): $N_{12}\propto \sqrt{s}$.

\hspace*{8.35cm}
\includegraphics*[scale=.35]{fig7}\hspace*{-3cm}
\vspace{-3.8cm}

Negatividad a campo nulo vs.\  $s$
para distintas anisotrop\'{\i}as {\color{blue}$\chi=\frac{J_y}{J_x}$}
\vspace*{-8.05cm}

\hspace*{10.7cm}\includegraphics*[scale=.3]{fig8}
\vspace*{1.cm}
%\begin{center}
%\includegraphics[width=0.8\linewidth]{placeholder}
%\captionof{figure}{Figure caption}
%\end{center}

%------------------------------------------------

%Nunc sit amet sem ut nulla tincidunt mattis vel nec mauris. Vestibulum odio tellus, lobortis. Vel %adipiscing, Aliquam dictum, ligula egestas commodo posuere, lectus lectus %congue ligula, sed posuere urna lectus at nisi. Aenean commodo risus ut dolor (viverra %scelerisque). Nullam varius, lacus et interdum hendrerit, odio orci ultrices mauris, id interdum eros mauris at urna. Fusce in nisi eros, sit amet volutpat turpis, \textbf{porttior magna} (commodo blandit euismod) \textbf{facilisis ornate magnis} (dis magnis). Aliquam ac justo lectus. Nunc ultrices aliquet purus non dictum. Nulla facilisi. Quisque vitae urna non purus sollicitudin venenatis. Aliquam erat volutpat. Cum sociis natoque penatibus et magnis dis parturient montes, nascetur ridiculus mus. In hendrerit tortor sed massa consequat eu viverra justo porta. Ut nec felis sem, non elementum.
}

%----------------------------------------------------------------------------------------

\end{poster}

\end{document}
